\documentclass{article}

%% Language and font encodings
\usepackage[utf8x]{inputenc}
\usepackage{polski}
\usepackage{hyperref}
\usepackage{dirtytalk}

%% Useful packages
\usepackage{amsmath}
\usepackage{amsthm}
\usepackage{amsfonts}
\usepackage[colorinlistoftodos]{todonotes}

\newtheorem{theorem}{Twierdzenie}
\newtheorem{definition}{Definicja}
\newtheorem{axiom}{Aksjomat}
\newtheorem{axiom-g}{Aksjomat}
\newtheorem{definition-g}{Definicja}
\newtheorem{theorem-g}{Twierdzenie}
\newtheorem{lemma}{Lemat}
\newtheorem{corollary}{Fakt}
\newtheorem*{axiom-a}{Aksjomat 1*}

\title{Od Anzelma do Gödla - dowód ontologiczny a logika modalna.}
\author{Artur M. Brodzki}

\begin{document}
\maketitle

\begin{abstract}
	W 1078 roku, Anzelm z Canterbury opublikował \emph{Proslogion}, w którym zawarł dowód na istnienie Boga oparty tylko i wyłącznie o definicję atrybutów boskości, znany powszechnie jako dowód ontologiczny. Jakkolwiek dowód Anzelma został odrzucony przez średniowieczną scholastykę po krytyce ze strony Gaunilona, nieoczekiwanie sformalizował go na gruncie nowoczesnej matematyki  Kurt Gödel. Jakkolwiek oryginalny zestaw aksjomatów Gödla okazał się być wewnętrznie sprzeczny i prowadził do zjawiska modalnego kolapsu, Curtis Anderson zmodyfikował pierwotną wersję, co pozwoliło na uniknięcie sprzeczności. Niezależnie od formalnej poprawności dowodu, trwają dyskusje na temat interpretacji uzyskanych wyników. Ponieważ przyjęta przez Gödla aksjomatyzacja pojęcia dobra i zła nie jest jednoznacznie określona, powstają pytania o dokładną tożsamość Boga, którego istnienia dowodzi dowód ontologiczny - jak i o możliwości uniknięcia tej wieloznaczności. 
\end{abstract}

\section{Wstęp} \label{sec:intro}

\subsection{Rola języka dla wymiany myśli} \label{sec:language}

Rozmowa i dialog wymagają zawsze na początku ustalenia wspólnego dla rozmówców języka. Język to symboliczna reprezentacja świata materialnego. Symbolom języka - głoskom, słowom i zdaniom odpowiadają obiekty, procesy czy stany świata, o którym ten język mówi. Można powiedzieć, że z matematycznego punktu widzenia język i jego interpretacja tworzą - przynajmniej powinny tworzyć - bijekcję, czyli odwzorowanie wzajemnie jednoznaczne. Gdy obaj rozmówcy posługują się w dyskusji tą samą bijekcją, wówczas rozmowa spełnia swoją rolę komunikacyjną i pozwala na wymianę myśli. Języki zwykle wyróżniamy nieco demograficznie: mówimy o języku polskim, niemieckim, białoruskim, chińskim - i zakładamy, że dwie osoby posługujące się "tym samym" językiem np. polskim, są w stanie wzajemnie się porozumieć, bo posługują się tymi samymi słowami. O tym, jak bardzo jest to błędne założenie, przekonać się można śledząc dyskusje i spory toczące się na forach i portalach internetowych. Uważna lektura tych rozmów pokazuje, że bardzo często długie i nieraz emocjonalne spory wynikają głównie z innej interpretacji tych samych wyrazów. Dopóki rozmówcy nie zauważą, że tych samych słów używają w zupełnie innych znaczeniach, jakiekolwiek porozumienie ani w ogóle racjonalna dyskusja po prostu nie jest możliwa. 

Jest to o tyle ciekawe, że zjawisko to jest powszechnie niedoceniane, często niemal niezauważane, tymczasem jego występowanie powinno być oczywiste, biorąc pod uwagę proces uczenia się języka przez ludzi. Języka uczymy się jako kilkuletnie dzieci, obserwując i słuchając dorosłych posługujących się mową. Dziecko poznaje znaczenie słów - czyli wytwarza w swoim umyśle tę bijekcję, o której była mowa - poprzez zapamiętywanie i uogólnianie osobistych doświadczeń. Niektóre słowa opisują konkretne, rzeczywiste obiekty jak stół, samochód czy pies - i ich znaczenie łatwo jest jednoznacznie ustalić. Niektóre jednak są bardziej abstrakcyjne, a często niedookreślone ze znaczeniem zależnym od kontekstu. Dotyczy to pojęć takich jak dobro, zło, wina, przyczyna, odpowiedzialność \footnote{Różne pojmowanie znaczenia tego ostatniego słowa uwidacznia się szczególnie często w dyskusjach na temat przemocy seksualnej. }. Te pojęcia należą do najczęściej wykorzystywanych, a przy tym (być może właśnie dlatego )najbardziej wieloznacznych i kontekstowych. Trudno oczekiwać, że każdy z 7 miliardów ludzi wytworzy w swoim umyśle dokładnie te same ich interpretacje, obserwując jedynie codzienne sytuacje i interakcje dorosłych w swoim rodzinnym domu. Właściwie można pokusić się o stwierdzenie, że w pewnym zakresie każdy człowiek posiada i tworzy swój własny, indywidualny język, niepodobny i nie do końca kompatybilny z językiem jakiejkolwiek innej osoby na świecie. 

\subsection{Spory o definicję i racjonalność pojęcia Boga. } \label{sec:god-def}

Problem języka nie ominął nauki, filozofii i religii, a może nawet w tych właśnie dziedzinach uwidocznił się szczególnie mocno. Niektórzy wręcz twierdzą za Wittgensteinem, że wszystkie wielkie i nierozwiązane spory filozofii jak problem uniwersaliów stanowią jedynie pewne gry językowe i wynikają z niedoprecyzowania stosowanych pojęć jak i samego pytania. Inni jednak (jak Popper) odrzucają tak radykalny pogląd. Niezależnie od tego, ustalenie wspólnych definicji podstawowych pojęć stanowi warunek racjonalnej dyskusji. 

Tak właśnie jest z niezwykle pojemnym i zmiennym pojęciem - czy obrazem - Boga. Dla starożytnych, potrzeba poznania Boga, czy częściej bogów, wynikała zazwyczaj z doświadczenia własnej słabości i kruchości wobec sił natury. Bogowie odpowiadali najgroźniejszym albo najbardziej zmiennym zjawisko, jak Słońce, morze czy dzikie zwierzęta. Do bogów modlono się w celu ich przebłagania, a relacja człowieka i bóstwa miała charakter myślenia magicznego i zaklęć mających dać człowiekowi kontrolę nad otoczeniem. Rolą figury boga było zmniejszenie lęku wynikającego z doświadczanego niebezpieczeństwa. 

Obraz bogów ulegał jednak z czasem sublimacji. Bogowie zaczęli reprezentować nie tylko zagrożenia, ale również idee i cnoty: grecka Atena odpowiadała za wiedzę i mądrość, 9 muz za piękno i sztukę. Dalszej modyfikacji dokonali greccy filozofowie: Sokrates zakwestionował niepodważalną prawdziwość mitologii bogów olimpijskich, Platon natomiast rozważał boskość w odniesieniu do swojej teorii Idei. Idee były bytami wyższymi i doskonalszymi niż świat materialny, a spośród idei najwyższa była Forma Dobra: doskonała wiedza i prawda, określająca granicę pomiędzy dobrem a złem, pierwsza przyczyna, dla której istnieją wszystkie pozostałe Idee i byty. Późniejsi filozofowie krytykowali platońską wersję Idei Dobra jako zbyt abstrakcyjną lub zbyt ogólną, niemniej pozostała ona znana i wywierała wpływ na myśl późniejszą, w tym szczególnie na teologię chrześcijańską. 

W chrześcijaństwie Bóg jest przede wszystkim stwórcą wszechświata, jako jedyny posiada zdolność stwarzania bytów z nicości, tzw. \emph{creatio ex nihilo}. Jest również najwyższym (a często, np. u Augustyna - jedynym) dobrem i celem wszelkiego istnienia. W tym sensie można go utożsamiać z platońską Formą Dobra i taką interpretację przyjmowała część chrześcijańskich filozofów i teologów. Podobne podejście przyjął Anzelm, biskup Canterbury, kiedy tworzył swój dowód ontologiczny istnienia Boga; omówimy go szczegółowo w rozdziale \ref{sec:anzelm}. Oprócz pojmowania Boga jako stworzyciela i wyznacznika moralnego dobra, Bóg jest pierwszą przyczyną istnienia świata. Jako taki występuje u Tomasza z Akwinu oraz u Gottfrieda Leibniza, który ubrał ten pogląd w często powtarzaną maksymę \say{dlaczego istnieje raczej coś niż nic}. 

To ostatnie podejście jest szczególnie interesujące z punktu widzenia relacji pomiędzy Bogiem czy religią w ogólności a współczesną nauką. Metoda naukowa przyjmuje jako podstawę zbiór założeń określanych jako metodologiczny naturalizm. W skrócie, składa się on z następujących twierdzeń:

\begin{enumerate}
	\item \label{metodological:start} Istnieje obiektywna rzeczywistość zewnętrzna, której obraz jest współdzielony przez wszystkich obserwatorów. Świat nie jest jedynie wytworem ludzkiego umysłu (solipsyzm). 
	\item Świat podlega stałym, niezmiennym w czasie i przestrzeni zasadom znanym jako \emph{prawa przyrody}. 
	\item Prawa przyrody wyjaśniają kształt wszystkich zjawisk. 
	\item Ponieważ prawa przyrody są niezmienne w czasie i przestrzeni, rządzone nimi zjawiska zachowują się w sposób powtarzalny. 
	\item \label{metodological:end} Dzięki temu, można odkrywać kształt tych praw za pomocą metody eksperymentalnej. 
\end{enumerate}

Naturalizm metodologiczny jest założeniem, na którym opiera się metoda nauk przyrodniczych i jednocześnie wyjaśnia przyczynę skuteczności metody naukowej w poznawaniu świata. Naturalizm wyznacza jednak również dla niej granicę stosowalności: hipotetyczne zjawiska, które nie spełniają jednego z powyższych założeń (np. nie są powtarzalne w zadanych warunkach) nie dają się badać za pomocą metody naukowej.

Naturalizm bywa jednak interpretowany nie jako założenie wyznaczające granicę metody naukowej, ale jako fakt dotyczący całej rzeczywistości. W tej wersji, którą nazywał będę naturalizmem filozoficznym, za prawdziwe uznaje się jedynie zjawiska, które da się zbadać i udowodnić za pomocą metody naukowej. Zjawiska nie spełniające potencjalnie założeń \ref{metodological:start} - \ref{metodological:end} odrzuca się jako nieprawdziwe z definicji \footnote{A przynajmniej traktuje się je jako zjawiska, o których nie da się racjonalnie dyskutować, bo są niepoznawalne - zgodnie z radykalną interpretacją zasady brzytwy Ockhama}. Tak rozumiany naturalizm stanowi podstawę tzw. \emph{nowego ateizmu}, reprezentowanego głównie przez tzw. Czterech Jeźdźców: Richarda Dawkinsa, Daniela Dennetta, Sama Harrisa, Christophera Hitchhensa. 

Takie podejście nie wydaje się jednak autorowi przekonujące. Założenia \ref{metodological:start} - \ref{metodological:end} mogą być postrzegane jako dość oczywiste i w zdecydowanej większości sytuacji okazują się skuteczne - mają jednak charakter właściwie metafizyczny i są niedowodliwe na gruncie samej metody naukowej. Tak restrykcyjne kryteria prawdziwości okazują się być sprzeczne ze sobą. Jest to szczególny przypadek problemów doświadczanych wcześniej przez pozytywizm: po zastosowaniu pozytywistycznych kryteriów prawdy do samego pozytywizmu, okazuje się, że pozytywizm sam wyprowadza swoją własną fałszywość. Jeżeli więc pozytywizm i jego pochodne były sposobem na ucieczkę od wszelkiej metafizyki postrzeganej jako nieracjonalna, to próba ta udowodniła, że od wszelkiej metafizyki uciec się nie da. A skoro tak, to zasadne jest zadawanie metafizycznych pytań i szukanie odpowiedzi - i traktowanie tych pytań jako racjonalnych. Racjonalne jest pytanie o istnienie Boga i racjonalna jest analiza logiczna dowodów mających istnienie Boga uzasadniać. Tutaj skupimy się na dowodzie ontologicznym Anzelma, który został sformalizowany na gruncie nowoczesnej matematyki - konkretnie logiki modalnej - przez Kurta Gödla. 

W następnym rozdziale opiszę dokładniej strukturę oryginalnego dowodu Anzelma oraz jego średniowieczną krytykę. Rozdział \ref{sec:modal-logic} wprowadza podstawy logiki modalnej jako teorii matematycznej, a w rozdziale \ref{sec:godel-proof} omówię dowód wprowadzony przez Kurta Gödla, a w \ref{sec:anderson-proof} - jego wady oraz dalsze modyfikacje. 

\section{Dowód Anzelma} \label{sec:anzelm}

Autorem dowodu ontologicznego jest biskup Anzelm z Canterbury, który opublikował go po raz pierwszy w 1078 roku w \emph{Proslogionie}. Argument Anzelma kształtował się w tle sporu, który szczególnie gorąco zajmował średniowiecznych teologów: sporu o uniwersalia. 

Spór ten rozpoczął się dla średniowiecznej filozofii w zasadzie od Boecjusza, który przetłumaczył na łacinę księgi Porfiriusza, greckiego filozofa z przełomu III w IV wieku naszej ery. Porfiriusz, jako neoplatończyk, rozważał teorię form, zadając do niej dwa fundamentalne pytania:
\begin{itemize}
	\item Czy platońskie idee - uniwersalia - istnieją tylko w umyśle, jako wytwory ludzkiej myśli, czy też istnieją realnie - niezależnie od umysłu?
	\item Jeśli istnieją realnie, to czy są bytem fizycznym i jak są powiązane z konkretnymi obiektami fizycznymi, które reprezentują?
\end{itemize}

Spór nie został ostatecznie rozwiązany w średniowieczu, co wynika zapewne ze wspomnianej na wstępie kwestii nieprecyzyjnego języka, i w różnych odmianach toczy się w filozofii do dzisiaj. W takiej atmosferze pracował i pisał Anzelm: tworząc swój dowód, chciał odnieść się do zarzutu, że idea Boga została stworzona przez człowieka i istnieje tylko i wyłącznie w jego ułomnym umyśle. Bardzo przewidująco zresztą postąpił, bo przecież podobne zarzuty pojawiają się wobec teizmu do dzisiaj. 

Kolejną istotną cechą dowodu Anzelma jest to, że z zaskakującą konsekwencją korzysta z myślenia opartego na wnioskowaniu z przyjętych na początku aksjomatów. Taki sposób myślenia stanowi dzisiaj podstawę wszelkiej matematyki; to właśnie aksjomatyczna struktura dowodu Anzelma przesądziła o tym, że tak łatwo daje się on przenieść na grunt nowoczesnej nauki. Niezależnie od ostatecznej oceny prawdziwości dowodu ontologicznego, w opinii autora pozostaje on jedną z najbardziej ponadczasowych konstrukcji myślowych stworzonych przez średniowieczną scholastykę. 

Anzelm zaczyna swój dowód od uściślenia samego pojęcia boskości. Najpierw jednak czyni pewne założenia wstępne: 

\begin{axiom} \label{axiom:1}
	Wszystkim istniejącym bytom można przypisać cechę \emph{doskonałości}. Różne byty posiadają cechę doskonałości w róznym stopniu. 
\end{axiom}

Doskonałość można tutaj rozumieć w sensie moralnym, jak i jako piękno czy użyteczność. Dokładny sposób wartościowania obiektów pod względem tej cechy nie jest jednak dla Anzelma istotny; ważny dla dowodu ontologicznego okazuje się jedynie kolejny aksjomat:

\begin{axiom} \label{axiom:2}
	Byt istniejący obiektywnie jest bardziej doskonały, niż identyczny byt, ale istniejący tylko w ludzkim umyśle. 
\end{axiom}

Aksjomat \ref{axiom:2} pozwala na ustanowienie pewnej hierarchii bytów, dzieląc je na mniej lub bardziej doskonałe. Mając już ustalony pewien aparat pojęciowy, mógł Anzelm przystąpić do zdefiniowania samego pojęcia Boga:

\begin{definition} \label{def:god}
	Bóg jest to byt, od którego nie ma (wręcz nie można sobie wyobrazić) żadnego bytu bardziej doskonałego. 
\end{definition}

Na bazie podanej przez siebie definicji, daje się już udowodnić twierdzenie o tym, że Bóg istnieje:

\begin{theorem} \label{theorem:god}
	Bóg jest bytem istniejącym realnie, poza ludzkim umysłem. 
\end{theorem}

\begin{proof}
	Dowód twierdzenia odbywa się jako dowód przez zaprzeczenie. Załóżmy, że Bóg istnieje tylko jako wytwór myśli człowieka. Wynika z tego, że nie jest to idea najdoskonalsza ze wszystkich, można bowiem wyobrazić sobie Boga bardziej doskonałego: takiego, który istnieje w realnej rzeczywistości. Wniosek ten jest jednak sprzeczny z definicją \ref{def:god}, zgodnie z którą nie ma żadnej doskonalszej od Boga idei. Uznając założenie początkowe za prawdziwe, otrzymujemy sprzeczność - a zatem Bóg musi być bytem istniejącym realnie. 
\end{proof}
	
Dowód Anzelma, jakkolwiek niezwykle wyrafinowany w formie, spotkał się z krytyką i to niemal natychmiast po opublikowaniu. Najwyżej 5 lat po wydaniu \emph{Proslogionu} \footnote{Gaunilon zmarł w 1083 roku.}, pojawiła się krótka rozprawka Gaunilona, benedyktyna z Marmoutier, zatytułowana \emph{W obronie głupiego}. W rozprawce tej Gaunilon zwraca uwagę na fakt, że korzystając z metody Anzelma udowodnić można niepokojąco wiele i podaje sławny do dziś przykład idealnej wyspy. Jeżeli - jak twierdzi Anzelm - byt istniejący realnie jest z definicji bardziej doskonały od istniejącego w umyśle, to istnieć musi gdzieś na świecie idealna, pozbawiona jakichkolwiek wad wyspa - raj. Taka wyspa posiadać musi bowiem wszystkie możliwe dla wysp przymioty, w szczególności zaś nie może jej brakować Anzelmowskiego przymiotu realnego istnienia. Rozumowanie to można rozciągnąć na bardzo wiele teologicznie niepożądanych bytów: herosów czy konkurujących z chrześcijańskim Bogiem półbogów, którzy nie posiadają wszystkich możliwych zalet, nie brakuje im jednak cechy realnego istnienia. 

Krytyka Gaunilona została powszechnie zaakceptowana a sam dowód ontologiczny był przez scholastykę zazwyczaj odrzucany, a czasem nawet postrzegany jako przejaw pewnej pychy ludzkiego rozumu, który rości sobie prawo do poznania natury i istoty Boga. Niespodziewanie zyskał jednak ponowne zainteresowanie po jego przeformułowaniu przez Kurta Gödla w języku logiki modalnej. Gödlowski dowód pojawia się po raz pierwszy w jego osobistych zapiskach, datowanych na rok 1941, on sam jednak nie wspomniał o nim nikomu aż do 1970 roku. Wspomnienia Oskara Morgensterna sugerują, że Gödel zwlekał z publikowaniem swoich wyników ze względów kulturowych. Prace na ten temat przekazał jednak w 1970 roku Dana Scottowi \footnote{Dana Scott pozostaje znany w środowisku informatyków głównie z prac dotyczących semantyki języków programowania, które posłużyły później w budowie nowoczesnych języków programowania funkcyjnego, a zwłaszcza Haskella} i ostatecznie zostały one opublikowane z 1987 roku, w 9 lat po jego śmierci \cite{godel1995}. 

\section{Logika modalna} \label{sec:modal-logic}

Aby omówić dowód Gödla w szczegółach, musimy jednak najpierw zapoznać się nieco z podstawami logiki modalnej, w której Gödel zapisał swój dowód ontologiczny. Logika modalna należy do logik nieklasycznych, tzn. różni się od rachunku zdań, stworzonego jeszcze przez Arystotelesa w jego sześciotomowym \emph{Organonie}. Logika modalna wprowadza do rachunku zdań dwa dodatkowe spójniki, tzw. spójniki modalne:
\begin{enumerate}
	\item Spójnik możliwości $\diamondsuit$
	\item Spójnik konieczności $\Box$
\end{enumerate}

Spójniki te posiadają nazwy zgodne z ich intuicyjnym znaczeniem, jednak do ich semantyki wrócimy szczegółowo za chwilę. Istotna jest aksjomatyka dotycząca ich wzajemnych relacji. Przypomina ona trochę klasyczne prawa De Morgana dla kwantyfikatorów: 

\begin{enumerate}
	\item Zdanie $Z$ jest niemożliwe wtedy i tylko wtedy, gdy musi być nieprawdziwe:
	\begin{equation} \label{eq:de-morgan1}
	\neg \left( \diamondsuit Z \right) \Leftrightarrow \Box \left( \neg Z \right)
	\end{equation}
	\item Zdanie $Z$ jest niekonieczne, wtedy i tylko wtedy, gdy może być nieprawdziwe:
	\begin{equation} \label{eq:de-morgan2}
	\neg \left( \Box Z \right) \Leftrightarrow \diamondsuit \left( \neg Z \right)
	\end{equation}
\end{enumerate}

Wykonując obustronną negację równań \ref{eq:de-morgan1} - \ref{eq:de-morgan2} można je przedstawić w równoważnej formie: 

\begin{enumerate}
	\item Zdanie $Z$ jest możliwe wtedy i tylko wtedy, gdy nie musi być nieprawdziwe:
	\begin{equation} \label{eq:de-morgan3}
	\diamondsuit Z \Leftrightarrow \neg \ \Box \left( \neg Z \right)
	\end{equation}
	\item Zdanie $Z$ jest konieczne wtedy i tylko wtedy, gdy nie może być nieprawdziwe:
	\begin{equation} \label{eq:de-morgan4}
	\Box Z \Leftrightarrow \neg \ \diamondsuit \left( \neg Z \right)
	\end{equation}
\end{enumerate}

Za pomocą logiki modalnej można wyrażać stwierdzenia charakteryzujące się różnym stopniem pewności:
\begin{itemize}
	\item \emph{Jutro nie musi padać. }
	\item \emph{Możliwe, że ustawa zostanie uchwalona. }
	\item \emph{Z pewnością poniesie on tego konsekwencje}
\end{itemize}
Korzystając z równań - aksjomatów \ref{eq:de-morgan1} - \ref{eq:de-morgan4} możemy przekształcić powyższe zdania w ich równoważne formy:
\begin{itemize}
	\item \emph{Możliwe, że jutro nie będzie padać. }
	\item \emph{Ustawa nie musi zostać nieuchwalona. }
	\item \emph{Niemożliwe, żeby nie poniósł on tego konsekwencji. } 
\end{itemize}

Intuicyjnie łatwo jest zrozumieć semantykę logiki modalnej, można ją jednak bardziej sformalizować, wykorzystując tzw. relację dostępności $\subset_R$. Zakładamy, że dany jest pewien stan rzeczy, tzw. \emph{świat} $S$. Z obecnego stanu rzeczy $S$ moga wyniknąć różne potencjalne możliwości, potencjalne scenariusze rozwoju wydarzeń. Scenariusze $p$, które mogą rozwinąć się ze świata $S$ pozostają z nim w relacji dostępności: $p \subset_R S$. Relacja $\subset_R$ nie jest z góry określona w sposób jednoznaczny, jednak gdy jest już dana, można za jej pomocą określić semantykę spójników $\diamondsuit$ i $\Box$:
\begin{enumerate}
	\item Zdanie $Z$ jest możliwe w świecie $S$, gdy istnieje co najmniej jeden świat $p \subset_R S$ w którym jest ono prawdziwe:
	\begin{equation}
	\diamondsuit Z \Leftrightarrow \exists p: p \subset_R S
	\end{equation}
	\item Zdanie $Z$ jest konieczne w świecie $S$, gdy w każdym świecie $p \subset_R S$ jest ono prawdziwe:
	\begin{equation}
	\Box Z \Leftrightarrow \forall p: p \subset_R S
	\end{equation}
\end{enumerate}
Z tak zdefiniowanej semantyki spójników modalnych wynika też natychmiast, dlaczego zakłada się, że ich wzajemna relacja ma charakter syntaktycznie zgodny z prawami De Morgana dla kwantyfikatorów. 

Prześledźmy działanie relacji dostępności na przykładzie. Niech $S$ oznacza świat w dniu dzisiejszym, powiedzmy, że jest to poniedziałek, a my zastanawiamy się nad jutrzejszą pogodą. Z $S$ mogą wyniknąć różne stany światy we wtorek: ponieważ nie wiemy nic o pogodzie na jutro, zakładamy, że może padać: $p_1 \subset_R S$ jak i być słonecznie $p_2 \subset_R S$. Zatem z $S$ dostępne są co najmniej dwa "światy": \say{pada} $p_1$ i \say{jest słonecznie} $p_2$. Możemy teraz wypowiedzieć się o jutrzejszej pogodzie używając spójników modalnych. Zdanie $Z_1 =$\emph{jutro będzie padać} jest prawdziwe w $p_1$, ale fałszywe w $p_2$, użyjemy tu więc spójnika możliwości: $\diamondsuit Z_1$. To samo dotyczy zdania $Z_2 =$\emph{jutro będzie słonecznie}: $\diamondsuit Z_2$. Możemy jednak nadać inny kształt relacji dostępności. Być może oglądaliśmy prognozę pogody na jutro i wiemy (z - jak zawsze - bardzo dużym stopniem pewności) że we wtorek spadnie 5 mm deszczu. Wtedy z $S$ dostępny jest już tylko świat $p_1$, ale nie $p_2$ i użyte przez nas spójniki modalne muszą się zmienić. Zdanie $Z_1$ jest teraz prawdziwe w każdym $p \subset_R S$ (bo jedynym takim $p$ jest $p_1$), co wypełnia definicję spójnika konieczności: $\Box Z_1$. Z kolei zdanie $Z_2$ nie jest prawdziwe w żadnym $p \subset_R S$, czyli - używając kwantyfikatorów - powiemy, że $\neg\ \exists p \subset_R S: Z_2\ zachodzi\ w\ p$, a słoneczna pogoda jest jutro niemożliwa: $ \neg\ \diamondsuit Z_2$. 

Relacja dostępności, jak każda relacja w sensie matematycznym, może posiadać określone własności, takie jak refleksyjność czy zwrotność. Zależy to od sposobu jej zdefiniowania w konkretnym przypadku; każda nowa postać relacji $\subset_R$ definiuje nową, inną logikę modalną oznaczaną z podaniem własności obowiązującej w niej relacji $\subset_R$. Zazwyczaj wyróżnia się 4 podstawowe typy logik modalnych:
\begin{itemize}
	\item Typ T - relacja $\subset_R$ jest refleksyjna: $\forall S: S \subset_R S$ - każdy świat jest dostępny sam z siebie. Refleksyjność gwarantuje, że każdy świat posiada przynajmniej jeden stan rzeczy, który jest z niego dostępny (samego siebie) co ułatwia interpretację kwantyfikatorów, na których oparte są spójniki modalne: rozpatrywanie kwantyfikatorów na zbiorze pustym jest mało intuicyjne i zwykle mało przydatne. W logice typu T można zatem wyróżnić dodatkowy aksjomat: $\Box z \Rightarrow z$. 
	\item Typ B - relacja $\subset_R$ jest symetryczna: $S_1 \subset_R S_2 \Leftrightarrow S_2 \subset_R S_1$. Warunek ten można wyrazić za pomocą aksjomatu "koniecznej możliwości": $z \Rightarrow \Box \diamondsuit z$. 
	\item Typ 4 - relacja $\subset_R$ jest przechodnia: $S_1 \subset_R S \wedge S_2 \subset_R S_1 \Rightarrow S_2 \subset_R S$. Takie logiki mogą posłużyć do modelowania łańcucha przyczyn i następstw. Daje nam to aksjomat modalny w formie $\Box z \Rightarrow \Box \Box z$. 
	\item Typ 5 - relacja $\subset_R$ jest euklidesowa: \\ $S_1 \subset_R S \wedge S_2 \subset_R S \Rightarrow S_1 \subset_R S_2 \wedge S_2 \subset_R S_1$. \\
	Taka postać relacji dostępności pozwala nam wykorzystywać aksjomat: $\diamondsuit z \Rightarrow \Box \diamondsuit z$: "jeśli coś jest możliwe, to jest koniecznie możliwe". 
\end{itemize}
Wymienione powyżej typy logik modalnych i ich aksjomaty były wykorzystywane w dowodzie ontologicznym Gödla jak i jego modyfikacjach i w dowodzeniu jego własności. 

\section{Dowód ontologiczny Gödla} \label{sec:godel-proof}

Poznaliśmy już potrzebne nam podstawy logiki modalnej, możemy więc przeanalizować dowód ontologiczny Gödla w jego pełnej postaci. Zakładamy najpierw - podobnie jak Anzelm - że obiekty $x$ posiadają cechy - predykaty $\varphi(x), \psi(x), \xi(x)$ w sensie logicznym - i że te predykaty dają się opisać jako "pozytywne" $P(\varphi)$ lub "negatywne" (ściślej: nieprawda-że-pozytywne): $\neg P(\psi)$. Wprowadzamy również symboliczne oznaczenie cechy boskości $G: G(x)$ oznacza zatem, że $x$ jest Bogiem. 

Definiujemy teraz kilka aksjomatów dotyczących cech pozytywnych i negatywnych. 
\begin{axiom-g} \label{axiom:godel1}
	Brak dobra jest zły i vice versa: 
	\begin{equation*}
	\neg P(\varphi) \Leftrightarrow P(\neg \varphi),\ \emph{równoważnie}\ P(\varphi) \Leftrightarrow \neg P(\neg \varphi)
	\end{equation*}
\end{axiom-g}
Z aksjomatu \ref{axiom:godel1} wynika, że każda cecha jest albo pozytywna, albo negatywna: nie może być opcji pośredniej. 
\begin{axiom-g} \label{axiom:godel2}
	Z dobra może wynikać tylko dobro. Dobro nie może pociągać za sobą żadnego zła: 
	\begin{equation*}
	\left( P(\varphi) \wedge \Box \forall x: \varphi(x) \Rightarrow \psi(x) \right) \Rightarrow P(\psi)
	\end{equation*}
\end{axiom-g}
\begin{axiom-g} \label{axiom:godel3}
	Cechy dobre są dobre w każdym możliwym świecie. Dobro jest absolutne:
	\begin{equation*}
	P(\varphi) \Rightarrow \Box P(\varphi)
	\end{equation*}
\end{axiom-g}
Powyższe aksjomaty oddają intuicje dotyczące cech dobrych (pozytywnych) i złych (negatywnych), przyjmowane zazwyczaj mniej lub bardziej świadomie przez większość ludzi. Ich dość ogólny charakter decyduje o sile dowodu ontologicznego, jednak również o jego słabościach, o czym szczegółowo opowiem w rozdziale \ref{sec:anderson-proof}. 

Następnie Gödel definiuje pojęcie Boga $G(x)$:
\begin{definition-g} \label{def:godel1}
	Bóg to obiekt posiadający wszystkie możliwe cechy pozytywne: 
	\begin{equation*}
	G(x) \Leftrightarrow \forall \varphi \left( P(\varphi) \Leftrightarrow \varphi(x) \right)
	\end{equation*}
\end{definition-g}
Mogłoby się wydawać oczywiste, że predykat $G$ jest pozytywny, jednak - zaskakująco - $P(G)$ nie wynika z aksjomatów \ref{axiom:godel1} - \ref{axiom:godel3}. Jest tak dlatego, że $G$ definiuje się poprzez kwantyfikator po predykatach, a zatem $G$ jest predykatem wyższego rzędu (o 1) od pozytywnych cech, które z sobie zawiera. Wprowadzamy zatem
\begin{axiom-g} \label{axiom:godel4}
	P(G)
\end{axiom-g}
Korzystając z aksjomatu \ref{axiom:godel1} możemy zauważyć, że Bóg nie może posiadać żadnej cechy negatywnej, co nie jest wprost powiedziane w definicji \ref{def:godel1}. Gdyby Bóg posiadał cechę $\psi: \neg P(\psi)$, to z aksjomatu \ref{axiom:godel1} wynika, że $P(\neg \psi)$. Ponieważ $\neg \psi$ jest cechą pozytywną, to Bóg nie posiada cechy $\psi$, co jest sprzeczne z założeniem że ją posiada - a zatem założenie to musi być fałszywe. 

Z tak zdefiniowanych założeń możemy już wyprowadzić kilka interesujących wyników. Przede wszystkim okazuje się, że dla każdej pozytywnej własności $\varphi$ możemy znaleźć przynajmniej jeden obiekt, który tę własność posiada. Mówimy, że każda pozytywna właściwość jest \say{potencjalnie egzemplifikowana} (ang. \emph{possibly exemplified}). 
\begin{theorem-g} \label{th:godel1}
	$P(\varphi) \Rightarrow \diamondsuit \exists x: \varphi(x)$
\end{theorem-g}
Dowód twierdzenia \ref{th:godel1} jest wyrafinowany i nie będziemy go tutaj szczegółowo przytaczać. Jego ogólna idea sprowadza się do faktu, że gdy zbiór obiektów, które posiadają własność $\varphi$ jest pusty, to fakt ten okazuje się być sprzeczny z aksjomatem \ref{axiom:godel2}. Wynika to z tego, że zdanie $x \in \emptyset$ jest zawsze fałszywe i można na jego podstawie udowodnić wszystko. Szczegółowa postać dowodu dostępna jest w \cite{Anderson1990}.  

Na podstawie twierdzenia \ref{th:godel1} możemy wykazać, że istnienie Boga jest faktem możliwym:
\begin{theorem-g} \label{th:godel2}
	$\diamondsuit \exists x: G(x)$
\end{theorem-g}
\begin{proof}
	Ponieważ $P(G)$, to z twierdzenia \ref{th:godel1} natychmiast wynika, że $\diamondsuit \exists x: G(x)$, c.b.d.u.
\end{proof}

Do dalszych rozważań potrzebujemy jeszcze jednej definicji. Gödel był pod wielkim wrażeniem filozofii Leibniza i uwidacznia się to w jego dowodzie ontologicznym. Definiuje on formalnie pojęcie esencji:
\begin{definition-g}
	Predykat $\varphi$ jest \emph{esencją} $x$, gdy wynikają z niego wszystkie własności obiektu $x$:
	\begin{equation*}
	\varphi\ \emph{ess}\ x \Leftrightarrow \varphi(x) \wedge \forall \psi: \left( \psi(x) \Rightarrow \Box \forall y: \left( \varphi(y) \Rightarrow \psi(y) \right) \right)
	\end{equation*}
\end{definition-g}
Nietrudno zauważyć następujący fakt:
\begin{corollary}
	$G(x) \Rightarrow G\ \emph{ess}\ x$
\end{corollary}
\begin{proof}
	Bóg z definicji posiada wszystkie możliwe pozytywne cechy, a ponieważ - jak wspomnieliśmy wcześniej - Bóg nie może posiadać żadnej cechy negatywnej, to jego cechy pozytywne są jego wszystkimi cechami, a boskość jest jego esencją. 
\end{proof}
Przechodzimy teraz do formalizacji kluczowej części dowodu Anzelma: założenia, że obiekt istniejący realnie jest bardziej doskonały od identycznego obiektu, ale istniejącego tylko w ludzkim umyśle. W tym celu potrzebujemy jeszcze jednej definicji:
\begin{definition-g}
	Obiekt $x$ istnieje w sposób konieczny $E(x)$, jeśli dla każdej esencji $\psi$ obiektu $x$ istnieje co najmniej jeden obiekt posiadający cechę $\psi$:
	\begin{equation*}
	E(x) \Leftrightarrow \forall \psi: \left( \psi\ \emph{ess}\ x \Rightarrow\Box\ \exists x: \psi(x) \right)
	\end{equation*}
\end{definition-g}
Zgodnie z rozumowaniem Anzelma, wprowadzamy aksjomat, że \emph{istnienie konieczne} $E$ jest cechą pozytywną:
\begin{axiom-g}
	P(E)
\end{axiom-g}
Ponieważ $E(x)$ jest cechą pozytywną, a $G$ jest jedyną esencją Boga, to uzyskujemy natychmiastowy wniosek:
\begin{theorem-g}
	$\Box\ \exists x: G(x)$
\end{theorem-g}
\begin{proof}
	Bóg $x: G(x)$ posiada każdą cechę pozytywną, w szczególności zatem zachodzi $E(x)$. Z definicji predykatu $E$ mamy: 
	\begin{equation*}
	 \forall \psi: \left( \psi\ \emph{ess}\ x \Rightarrow\Box\ \exists x: \psi(x) \right)
	\end{equation*}
	A ponieważ $G\ \emph{ess}\ x$, to:
	\begin{equation*}
	\Box\ \exists x: G(x)
	\end{equation*}
	c.b.d.u.
\end{proof}
Udowodniliśmy, że Bóg istnieje w świecie w sposób konieczny. 

\section{Słabości dowodu i jego modyfikacje} \label{sec:anderson-proof}
Dowód ontologiczny Gödla wydaje się być formalnie poprawny, ale nie uniknął on licznych problemów. Pojawiają się zarzuty zarówno natury interpretacyjnej, jak i czysto formalnej. 

Gödel, w przeciwieństwie do Anzelma, nie czyni rozróżnienia na istnienie realne oraz w ludzkim umyśle. Uwidacznia się tutaj wspomniana we wstępie kwestia językowa - nie jest jasne, co tak naprawdę oznacza kwantyfikator $\exists$. Kwestią sporną jest również, czy w ogóle "istnienie" może być traktowane jako predykat; o ile bowiem o istniejącemu obiektowi można sensownie przypisać posiadanie pewnej cechy lub jej brak, o tyle nie jest jasne, czy można mówić o cechach - predykatach na obiektach nieistniejących. Do twierdzenia Gödla można stosować zarzut o dowodzeniu istnienia Boga jako idei matematycznej. Dowód Gödla mówi o Bogu, nie precyzuje jednak wprost jego cech - nie ma więc pewności, jakiego Boga rozpatruje się w dowodzie ontologicznym. 

Wiąże się z tym również kwestia dość ogólnej aksjomatyzacji pojęcia cech pozytywnych i negatywnych. Istnieje wiele możliwych sposobów przypisania konkretnym cechom rzeczywistych obiektów predykatu $P$ i każde z nich może spełniać aksjomaty \ref{axiom:godel1} - \ref{axiom:godel3}. Często klasyfikacja pewnych zjawisk jako pozytywnych lub negatywnych jest wśród ludzi kontrowersyjna. Boga której religii dowodzi więc dowód Gödla? A może każdemu wartościowaniu odpowiada osobny byt, reprezentujący Boga w danym, określonym systemie wartości? Twierdzenie nie pozwala uniknąć tej wieloznaczności. 

Przede wszystkim jednak dowód w wersji podanej w rozdziale \ref{sec:godel-proof} okazuje się być podatny na obiekcje typu Gaunilona - identyczne rozumowanie można zastosować do dowodzenia istnienia właściwie każdego możliwego bytu. W istocie prawdziwa jest mocniejsza hipoteza: zestaw aksjomatów \ref{axiom:godel1} - \ref{axiom:godel3} jest wewnętrznie sprzeczny; prowadzi on do zjawiska tzw. modalnego kolapsu, a na jego bazie można udowodnić wszystko. Po raz pierwszy sugestię o niespójności Gödlowskich aksjomatów wysunął Jordan Sobel w 1987 roku \cite{sobel1987}, pokazując, że z założeń Gödla wynika zdanie $z\Rightarrow\Box z$. Ostatecznie hipotezę Sobela udowodniono w 2016 z wykorzystaniem komputerowego wspomagania dowodzenia twierdzeń \cite{Benzmuller2016}. Jedynym warunkiem jest refleksyjna lub symetryczna relacja dostępności (logiki modalne typu T lub B); wówczas z aksjomatów  \ref{axiom:godel1} - \ref{axiom:godel3} daje się wyprowadzić zdanie $\diamondsuit\ \Box\ \bot$ (możliwe, że konieczny jest fałsz). Wyniki Bezmüllera to jeden z najbardziej spektakularnych przykładów zastosowania sztucznej inteligencji do zagadnień metafizycznych; zmuszają jednak one do modyfikacji oryginalnej wersji dowodu Gödla. 

Okazuje się, że do uniknięcia powyższych problemów wystarczy zmodyfikować aksjomat \ref{axiom:godel1}. Po raz pierwszy taką modyfikację zaproponował Anderson \cite{Anderson1990}, na 26 lat przed pojawieniem się wyników Benzmüllera; 6 lat później pokazano, że taka modyfikacja pozwala na uniknięcie zarówno zastrzeżeń Sobela, jak i Gaunilona \cite{Anderson1996}. Aksjomat 1 w zmodyfikowanej wersji przyjmuje formę:
\begin{axiom-a}
	Jeżeli własność $\varphi$ jest pozytywna, to jej brak jest cechą negatywną.
	\begin{equation*}
	P(\varphi) \Rightarrow \neg P(\neg \varphi),\ \emph{równoważnie}\ P(\neg \varphi) \Rightarrow \neg P(\varphi)
	\end{equation*}
\end{axiom-a}
Uważny czytelnik z pewnością dostrzeże subtelną zmianę: obustronna równoważność $\Leftrightarrow$ została zamieniona na jednostronną implikację $\Rightarrow$. Pierwotnie każda cecha jak i jej zaprzeczenie posiadały jednoznacznie określone wartościowanie; jeżeli $\varphi$ była pozytywna, to $\neg \varphi$ była negatywna; jeżeli $\psi$ była negatywna, to $\neg \psi$ była cechą pozytywną. Teraz możliwy jest stan indyferentny; z negatywności $\varphi$ nie możemy wyciągać wniosków na temat wartościowania $\neg \varphi$; możemy jedynie stwierdzić, że brak cechy pozytywnej jest negatywny, ale nigdy na odwrót. 

Szczegółowe wykazanie, że Aksjomat 1* pozwala uniknąć modalnego kolapsu przekracza ramy tego krótkiego eseju; zainteresowanych czytelników odsyłam do wymienionej już literatury. Istotne jest, że modyfikacja Andersona nie rozwiewa wszystkich wątpliwości; niektórzy zwracają uwagę, że przyjęty zbiór założeń w zasadzie jest epistemologicznie bardzo blisko przyjęcia po prostu założenia, że istnienie Boga jest możliwe. Ponownie odzywa się tutaj kwestia językowa; nie jest jasne, jak oceniać "postęp epistemologiczny" wprowadzany za pomocą twierdzenia ani jak jednoznacznie określić, czy dane twierdzenie wprowadza nową jakość do posiadanej przez nas wiedzy. Czy stwierdzenie, które jest tautologią, naprawdę można określić jako wartość dodaną? A przecież ostatecznie, jak wiemy, wszystkie twierdzenia matematyczne są tylko i wyłącznie zbiorem niezwykle skomplikowanych tautologii. 

\section{Perspektywy}
Trwają wysiłki mające na celu sformalizowanie dowodu Gödla do postaci umożliwiającej jego weryfikację za pomocą komputera; pomimo sukcesu Benzmüllera w analizie zjawiska modalnego kolapsu wciąż nie udało się udowodnić jakiejkolwiek wersji dowodu ontologicznego w całości komputerowo. Zdaniem autora, szczególnie interesujące byłoby jednak usunięcie - a przynajmniej ograniczenie - liczby możliwych wartościowań spełniających aksjomatyzację cech pozytywnych i negatywnych. Można zauważyć, że w całkowicie deterministycznym wszechświecie liczba możliwych wartościowań gwałtownie spada, z uwagi na aksjomaty \ref{axiom:godel2} i \ref{axiom:godel3}. Być może dalsze badania pozwoliłyby stworzyć zestaw odporny zarówno na kolaps modalny, jak i na wieloznaczność interpretacji. 

\bibliographystyle{apalike}
\bibliography{teologia.bib}

\end{document}